\documentclass[11pt]{article}

\newcommand{\m}[1]{{\bf{#1}}}       % for matrices and vectors
\newcommand{\tr}{^{\sf T}}          % transpose

\topmargin 0in
\textheight 9in
\oddsidemargin 0pt
\evensidemargin 0pt
\textwidth 6.5in

%------------------------------------------------------------------------------
\begin{document}
%------------------------------------------------------------------------------

\title{CAMD Version 2.1 User Guide}
\author{Patrick R. Amestoy\thanks{ENSEEIHT-IRIT,
2 rue Camichel 31017 Toulouse, France.
email: amestoy@enseeiht.fr.  http://www.enseeiht.fr/$\sim$amestoy.}
\and Yanqing (Morris) Chen
\and Timothy A. Davis\thanks{
Dept.~of Computer and Information Science and Engineering,
Univ.~of Florida, Gainesville, FL, USA.
email: davis@cise.ufl.edu.
http://www.cise.ufl.edu/$\sim$davis.
This work was supported by the National
Science Foundation, under grants ASC-9111263, DMS-9223088, and CCR-0203270.
Portions of the work were done while on sabbatical at Stanford University
and Lawrence Berkeley National Laboratory (with funding from Stanford
University and the SciDAC program).  Ordering constraints added with
support from Sandia National Laboratory (Dept. of Energy).
}
\and Iain S. Duff\thanks{Rutherford Appleton Laboratory, Chilton, Didcot, 
Oxon OX11 0QX, England. email: i.s.duff@rl.ac.uk.  
http://www.numerical.rl.ac.uk/people/isd/isd.html.
This work was supported by the EPSRC under grant GR/R46441.
}}

\date{May 5, 2006}
\maketitle

%------------------------------------------------------------------------------
\begin{abstract}
CAMD is a set of ANSI C routines that implements the approximate minimum degree
ordering algorithm to permute sparse matrices prior to
numerical factorization.  Ordering constraints can be optionally provided.
A MATLAB interface is included.
\end{abstract}
%------------------------------------------------------------------------------

CAMD Version 2.1, Copyright\copyright 2006 by Timothy A.  Davis,
Yanqing (Morris) Chen,
Patrick R. Amestoy, and Iain S. Duff.  All Rights Reserved.
CAMD is available under alternate licences; contact T. Davis for details.

{\bf CAMD License:}
    Your use or distribution of CAMD or any modified version of
    CAMD implies that you agree to this License.

    This library is free software; you can redistribute it and/or
    modify it under the terms of the GNU Lesser General Public
    License as published by the Free Software Foundation; either
    version 2.1 of the License, or (at your option) any later version.

    This library is distributed in the hope that it will be useful,
    but WITHOUT ANY WARRANTY; without even the implied warranty of
    MERCHANTABILITY or FITNESS FOR A PARTICULAR PURPOSE.  See the GNU
    Lesser General Public License for more details.

    You should have received a copy of the GNU Lesser General Public
    License along with this library; if not, write to the Free Software
    Foundation, Inc., 51 Franklin St, Fifth Floor, Boston, MA  02110-1301
    USA

    Permission is hereby granted to use or copy this program under the
    terms of the GNU LGPL, provided that the Copyright, this License,
    and the Availability of the original version is retained on all copies.
    User documentation of any code that uses this code or any modified
    version of this code must cite the Copyright, this License, the
    Availability note, and "Used by permission." Permission to modify
    the code and to distribute modified code is granted, provided the
    Copyright, this License, and the Availability note are retained,
    and a notice that the code was modified is included.

{\bf Availability:}
    http://www.cise.ufl.edu/research/sparse/camd

{\bf Acknowledgments:}

    This work was supported by the National Science Foundation, under
    grants ASC-9111263 and DMS-9223088 and CCR-0203270, and by Sandia
    National Labs (a grant from DOE).
    The conversion to C, the addition of the elimination tree
    post-ordering, and the handling of dense rows and columns
    were done while Davis was on sabbatical at
    Stanford University and Lawrence Berkeley National Laboratory.
    The ordering constraints were added by Chen and Davis.

%------------------------------------------------------------------------------
\newpage
\section{Overview}
%------------------------------------------------------------------------------

CAMD is a set of routines for preordering a sparse matrix prior to
numerical factorization.  It uses an approximate minimum degree ordering
algorithm \cite{AmestoyDavisDuff96,AmestoyDavisDuff04}
to find a permutation matrix $\m{P}$
so that the Cholesky factorization $\m{PAP}\tr=\m{LL}\tr$ has fewer
(often much fewer) nonzero entries than the Cholesky factorization of $\m{A}$.
The algorithm is typically much faster than other ordering methods
and  minimum degree ordering
algorithms that compute an exact degree \cite{GeorgeLiu89}.
Some methods, such as approximate deficiency
\cite{RothbergEisenstat98} and graph-partitioning based methods
\cite{Chaco,KarypisKumar98e,PellegriniRomanAmestoy00,schu:01}
can produce better orderings, depending on the matrix.

The algorithm starts with an undirected graph representation of a
symmetric sparse matrix $\m{A}$.  Node $i$ in the graph corresponds to row
and column $i$ of the matrix, and there is an edge $(i,j)$ in the graph if
$a_{ij}$ is nonzero.
The degree of a node is initialized to the number of off-diagonal nonzeros
in row $i$, which is the size of the set of nodes
adjacent to $i$ in the graph.

The selection of a pivot $a_{ii}$ from the diagonal of $\m{A}$ and the first
step of Gaussian elimination corresponds to one step of graph elimination.
Numerical fill-in causes new nonzero entries in the matrix
(fill-in refers to
nonzeros in $\m{L}$ that are not in $\m{A}$).
Node $i$ is eliminated and edges are added to its neighbors
so that they form a clique (or {\em element}).  To reduce fill-in,
node $i$ is selected as the node of least degree in the graph.
This process repeats until the graph is eliminated.

The clique is represented implicitly.  Rather than listing all the
new edges in the graph, a single list of nodes is kept which represents
the clique.  This list corresponds to the nonzero pattern of the first
column of $\m{L}$.  As the elimination proceeds, some of these cliques
become subsets of subsequent cliques, and are removed.   This graph
can be stored in place, that is
using the same amount of memory as the original graph.

The most costly part of the minimum degree algorithm is the recomputation
of the degrees of nodes adjacent to the current pivot element.
Rather than keep track of the exact degree, the approximate minimum degree
algorithm finds an upper bound on the degree that is easier to compute.
For nodes of least degree, this bound tends to be tight.  Using the
approximate degree instead of the exact degree leads to a substantial savings
in run time, particularly for very irregularly structured matrices.
It has no effect on the quality of the ordering.

The elimination phase is followed by an
elimination tree post-ordering.  This has no effect on fill-in, but
reorganizes the ordering so that the subsequent numerical factorization is
more efficient.  It also includes a pre-processing phase in which nodes of
very high degree are removed (without causing fill-in), and placed last in the
permutation $\m{P}$ (subject to the constraints).
This reduces the run time substantially if the matrix
has a few rows with many nonzero entries, and has little effect on the quality
of the ordering.
CAMD operates on the
symmetric nonzero pattern of $\m{A}+\m{A}\tr$, so it can be given
an unsymmetric matrix, or either the lower or upper triangular part of
a symmetric matrix.

CAMD has the ability to order the matrix with constraints.  Each
node $i$ in the graph (row/column $i$ in the matrix) has a constraint,
{\tt C[i]}, which is in the range {\tt 0} to {\tt n-1}.  All nodes with
{\tt C[i] = 0} are
ordered first, followed by all nodes with constraint {\tt 1}, and so on.
That is, {\tt C[P[k]]} is monotonically non-decreasing as {\tt k} varies from
{\tt 0} to {\tt n-1}.  If {\tt C} is NULL, no
constraints are used (the ordering will be similar to AMD's ordering,
except that the postordering is different).
The optional {\tt C} parameter is also provided in the MATLAB interface,
({\tt p = camd (A,Control,C)}).

For a discussion of the long history of the minimum degree algorithm,
see \cite{GeorgeLiu89}.

%------------------------------------------------------------------------------
\section{Availability}
%------------------------------------------------------------------------------

CAMD Version 2.1 is available at http://www.cise.ufl.edu/research/sparse.
The Fortran version is available as the routine {\tt MC47} in HSL
(formerly the Harwell Subroutine Library) \cite{hsl:2002}. {\tt MC47} does
not include ordering constraints.

%------------------------------------------------------------------------------
\section{Using CAMD in MATLAB}
%------------------------------------------------------------------------------

To use CAMD in MATLAB, you must first compile the CAMD mexFunction.
Just type {\tt make} in the Unix system shell, while in the {\tt CAMD}
directory.  You can also type {\tt camd\_make} in MATLAB, while in the
{\tt CAMD/MATLAB} directory.  Place the {\tt CAMD/MATLAB} directory in your
MATLAB path.  This can be done on any system with MATLAB, including Windows.
See Section~\ref{Install} for more details on how to install CAMD.

The MATLAB statement {\tt p=camd(A)} finds a permutation vector {\tt p} such
that the Cholesky factorization {\tt chol(A(p,p))} is typically sparser than
{\tt chol(A)}.
If {\tt A} is unsymmetric, {\tt camd(A)} is identical to {\tt camd(A+A')}
(ignoring numerical cancellation).
If {\tt A} is not symmetric positive definite,
but has substantial diagonal entries and a mostly symmetric nonzero pattern,
then this ordering is also suitable for LU factorization.  A partial pivoting
threshold may be required to prevent pivots from being selected off the
diagonal, such as the statement {\tt [L,U,P] = lu (A (p,p), 0.1)}.
Type {\tt help lu} for more details.
The statement {\tt [L,U,P,Q] = lu (A (p,p))} in MATLAB 6.5 is
not suitable, however, because it uses UMFPACK Version 4.0 and thus
does not attempt to select pivots from the diagonal.
UMFPACK Version 4.1 in MATLAB 7.0 and later
uses several strategies, including a symmetric pivoting strategy, and
will give you better results if you want to factorize an unsymmetric matrix
of this type.  Refer to the UMFPACK User Guide for more details, at
http://www.cise.ufl.edu/research/sparse/umfpack.

The CAMD mexFunction is much faster than the built-in MATLAB symmetric minimum
degree ordering methods, SYMAMD and SYMMMD.  Its ordering quality is
essentially identical to AMD, comparable to SYMAMD, and better than SYMMMD
\cite{DavisGilbertLarimoreNg04}.

An optional input argument can be used to modify the control parameters for
CAMD (aggressive absorption, dense row/column handling, and printing of
statistics).  An optional output
argument provides statistics on the ordering, including an analysis of the
fill-in and the floating-point operation count for a subsequent factorization.
For more details (once CAMD is installed),
type {\tt help camd} in the MATLAB command window.

%------------------------------------------------------------------------------
\section{Using CAMD in a C program}
\label{Cversion}
%------------------------------------------------------------------------------

The C-callable CAMD library consists of seven user-callable routines and one
include file.  There are two versions of each of the routines, with
{\tt int} and {\tt long} integers.
The routines with prefix
{\tt camd\_l\_} use {\tt long} integer arguments; the others use
{\tt int} integer arguments.  If you compile CAMD in the standard
ILP32 mode (32-bit {\tt int}'s, {\tt long}'s, and pointers) then the versions
are essentially identical.  You will be able to solve problems using up to 2GB
of memory.  If you compile CAMD in the standard LP64 mode, the size of an
{\tt int} remains 32-bits, but the size of a {\tt long} and a pointer both get
promoted to 64-bits.

The following routines are fully described in Section~\ref{Primary}:

\begin{itemize}
\item {\tt camd\_order}
({\tt long} version: {\tt camd\_l\_order})
    {\footnotesize
    \begin{verbatim}
    #include "camd.h"
    int n, Ap [n+1], Ai [nz], P [n], C [n] ;
    double Control [CAMD_CONTROL], Info [CAMD_INFO] ;
    int result = camd_order (n, Ap, Ai, P, Control, Info, C) ;
    \end{verbatim}
    }
    Computes the approximate minimum degree ordering of an $n$-by-$n$ matrix
    $\m{A}$.  Returns a permutation vector {\tt P} of size {\tt n}, where
    {\tt P[k] = i} if row and column {\tt i} are the {\tt k}th row and
    column in the permuted matrix.
    This routine allocates its own memory of size $1.2e+9n$ integers,
    where $e$ is the number of nonzeros in $\m{A}+\m{A}\tr$.
    It computes statistics about the matrix $\m{A}$, such as the symmetry of
    its nonzero pattern, the number of nonzeros in $\m{L}$,
    and the number of floating-point operations required for Cholesky and LU
    factorizations (which are returned in the {\tt Info} array).
    The user's input matrix is not modified.
    It returns {\tt CAMD\_OK} if successful,
    {\tt CAMD\_OK\_BUT\_JUMBLED} if successful (but the matrix had unsorted
    and/or duplicate row indices),
    {\tt CAMD\_INVALID} if the matrix is invalid,
    {\tt CAMD\_OUT\_OF\_MEMORY} if out of memory.

    The array {\tt C} provides the ordering constraints.
    On input, {\tt C} may be null (to denote no constraints);
    otherwise, it must be an array size {\tt n}, with entries in the range
    {\tt 0} to {\tt n-1}.
    On output, {\tt C[P[0..n-1]]} is monotonically non-descreasing.  

\item {\tt camd\_defaults}
({\tt long} version: {\tt camd\_l\_defaults})
    {\footnotesize
    \begin{verbatim}
    #include "camd.h"
    double Control [CAMD_CONTROL] ;
    camd_defaults (Control) ;
    \end{verbatim}
    }
    Sets the default control parameters in the {\tt Control} array.  These can
    then be modified as desired before passing the array to the other CAMD
    routines.

\item {\tt camd\_control}
({\tt long} version: {\tt camd\_l\_control})
    {\footnotesize
    \begin{verbatim}
    #include "camd.h"
    double Control [CAMD_CONTROL] ;
    camd_control (Control) ;
    \end{verbatim}
    }
    Prints a description of the control parameters, and their values.

\item {\tt camd\_info}
({\tt long} version: {\tt camd\_l\_info})
    {\footnotesize
    \begin{verbatim}
    #include "camd.h"
    double Info [CAMD_INFO] ;
    camd_info (Info) ;
    \end{verbatim}
    }
    Prints a description of the statistics computed by CAMD, and their values.

\item {\tt camd\_valid}
({\tt long} version: {\tt camd\_valid})
    {\footnotesize
    \begin{verbatim}
    #include "camd.h"
    int n, Ap [n+1], Ai [nz] ;
    int result = camd_valid (n, n, Ap, Ai) ;
    \end{verbatim}
    }
    Returns {\tt CAMD\_OK} or {\tt CAMD\_OK\_BUT\_JUMBLED}
    if the matrix is valid as input to {\tt camd\_order};
    the latter is returned if the matrix has unsorted and/or duplicate
    row indices in one or more columns. 
    Returns {\tt CAMD\_INVALID} if the matrix cannot be passed to
    {\tt camd\_order}.
    For {\tt camd\_order}, the matrix must
    also be square.  The first two arguments are the number of rows and the
    number of columns of the matrix.  For its use in CAMD, these must both
    equal {\tt n}.

\item {\tt camd\_2}
({\tt long} version: {\tt camd\_l2})
    CAMD ordering kernel.  It is faster than {\tt camd\_order}, and
    can be called by the user, but it is difficult to use.
    It does not check its inputs for errors.
    It does not require the columns of its input matrix to be sorted,
    but it destroys the matrix on output.  Additional workspace must be passed.
    Refer to the source file {\tt CAMD/Source/camd\_2.c} for a description.

\end{itemize}

The nonzero pattern of the matrix $\m{A}$ is represented in compressed column
form.
For an $n$-by-$n$ matrix $\m{A}$ with {\tt nz} nonzero entries, the
representation consists of two arrays: {\tt Ap} of size {\tt n+1} and {\tt Ai}
of size {\tt nz}.  The row indices of entries in column {\tt j} are stored in
    {\tt Ai[Ap[j]} $\ldots$ {\tt Ap[j+1]-1]}.
For {\tt camd\_order},
if duplicate row indices are present, or if the row indices in any given
column are not sorted in ascending order, then {\tt camd\_order} creates
an internal copy of the matrix with sorted rows and no duplicate entries,
and orders the copy.  This adds slightly to the time and memory usage of
{\tt camd\_order}, but is not an error condition.

The matrix is 0-based, and thus
row indices must be in the range {\tt 0} to {\tt n-1}.
The first entry {\tt Ap[0]} must be zero.
The total number of entries in the matrix is thus {\tt nz = Ap[n]}.

The matrix must be square, but it does not need to be symmetric.
The {\tt camd\_order} routine constructs the nonzero pattern of
$\m{B} = \m{A}+\m{A}\tr$ (without forming $\m{A}\tr$ explicitly if
$\m{A}$ has sorted columns and no duplicate entries),
and then orders the matrix $\m{B}$.  Thus, either the
lower triangular part of $\m{A}$, the upper triangular part,
or any combination may be passed.  The transpose $\m{A}\tr$ may also be
passed to {\tt camd\_order}.
The diagonal entries may be present, but are ignored.

%------------------------------------------------------------------------------
\subsection{Control parameters}
\label{control_param}
%------------------------------------------------------------------------------

Control parameters are set in an optional {\tt Control} array.
It is optional in the sense that if
a {\tt NULL} pointer is passed for the {\tt Control} input argument,
then default control parameters are used.
%
\begin{itemize}
\item {\tt Control[CAMD\_DENSE]} (or {\tt Control(1)} in MATLAB):
controls the threshold for ``dense''
rows/columns.  A dense row/column in $\m{A}+\m{A}\tr$
can cause CAMD to spend significant time
in ordering the matrix.  If {\tt Control[CAMD\_DENSE]} $\ge 0$,
rows/columns with
more than {\tt Control[CAMD\_DENSE]} $\sqrt{n}$ entries are ignored during
the ordering, and placed last in the output order.  The default
value of {\tt Control[CAMD\_DENSE]} is 10.  If negative, no rows/columns
are treated as ``dense.''  Rows/columns with 16 or fewer off-diagonal
entries are never considered ``dense.''
%
\item {\tt Control[CAMD\_AGGRESSIVE]} (or {\tt Control(2)} in MATLAB):
controls whether or not to use
aggressive absorption, in which a prior element is absorbed into the current
element if it is a subset of the current element, even if it is not
adjacent to the current pivot element (refer
to \cite{AmestoyDavisDuff96,AmestoyDavisDuff04}
for more details).  The default value is nonzero,
which means that aggressive absorption will be performed.  This nearly always
leads to a better ordering (because the approximate degrees are more
accurate) and a lower execution time.  There are cases where it can
lead to a slightly worse ordering, however.  To turn it off, set
{\tt Control[CAMD\_AGGRESSIVE]} to 0.
%
\end{itemize}

Statistics are returned in the {\tt Info} array
(if {\tt Info} is {\tt NULL}, then no statistics are returned).
Refer to {\tt camd.h} file, for more details
(14 different statistics are returned, so the list is not included here).

%------------------------------------------------------------------------------
\subsection{Sample C program}
%------------------------------------------------------------------------------

The following program, {\tt camd\_demo.c}, illustrates the basic use of CAMD.
See Section~\ref{Synopsis} for a short description
of each calling sequence.

{\footnotesize
\begin{verbatim}
#include <stdio.h>
#include "camd.h"

int n = 5 ;
int Ap [ ] = { 0,   2,       6,       10,  12, 14} ;
int Ai [ ] = { 0,1, 0,1,2,4, 1,2,3,4, 2,3, 1,4   } ;
int C [ ] = { 2, 0, 0, 0, 1 } ;
int P [5] ;

int main (void)
{
    int k ;
    (void) camd_order (n, Ap, Ai, P, (double *) NULL, (double *) NULL, C) ;
    for (k = 0 ; k < n ; k++) printf ("P [%d] = %d\n", k, P [k]) ;
    return (0) ;
}

\end{verbatim}
}

The {\tt Ap} and {\tt Ai} arrays represent the binary matrix
\[
\m{A} = \left[
\begin{array}{rrrrr}
 1 &  1 &  0 &  0 &  0 \\
 1 &  1 &  1 &  0 &  1 \\
 0 &  1 &  1 &  1 &  0 \\
 0 &  0 &  1 &  1 &  0 \\
 0 &  1 &  1 &  0 &  1 \\
\end{array}
\right].
\]
The diagonal entries are ignored.
%
CAMD constructs the pattern of $\m{A}+\m{A}\tr$,
and returns a permutation vector of $(3, 2, 1, 4, 0)$.
Note that nodes 1, 2, and 3 appear first (they are in the constraint set 0),
node 4 appears next (since {\tt C[4] = 1}), and node 0 appears last.
%
Since the matrix is unsymmetric but with a mostly symmetric nonzero
pattern, this would be a suitable permutation for an LU factorization of a
matrix with this nonzero pattern and whose diagonal entries are not too small.
The program uses default control settings and does not return any statistics
about the ordering, factorization, or solution ({\tt Control} and {\tt Info}
are both {\tt (double *) NULL}).  It also ignores the status value returned by
{\tt camd\_order}.

More example programs are included with the CAMD package.
The {\tt camd\_demo.c} program provides a more detailed demo of CAMD.
Another example is the CAMD mexFunction, {\tt camd\_mex.c}.

%------------------------------------------------------------------------------
\subsection{A note about zero-sized arrays}
%------------------------------------------------------------------------------

CAMD uses several user-provided arrays of size {\tt n} or {\tt nz}.
Either {\tt n} or {\tt nz} can be zero.
If you attempt to {\tt malloc} an array of size zero,
however, {\tt malloc} will return a null pointer which CAMD will report
as invalid.  If you {\tt malloc} an array of
size {\tt n} or {\tt nz} to pass to CAMD, make sure that you handle the
{\tt n} = 0 and {\tt nz = 0} cases correctly.

%------------------------------------------------------------------------------
\section{Synopsis of C-callable routines}
\label{Synopsis}
%------------------------------------------------------------------------------

The matrix $\m{A}$ is {\tt n}-by-{\tt n} with {\tt nz} entries.

{\footnotesize
\begin{verbatim}
#include "camd.h"
int n, status, Ap [n+1], Ai [nz], P [n], C [n] ;
double Control [CAMD_CONTROL], Info [CAMD_INFO] ;
camd_defaults (Control) ;
status = camd_order (n, Ap, Ai, P, Control, Info, C) ;
camd_control (Control) ;
camd_info (Info) ;
status = camd_valid (n, n, Ap, Ai) ;
\end{verbatim}
}

The {\tt camd\_l\_*} routines are identical, except that all {\tt int}
arguments become {\tt long}:

{\footnotesize
\begin{verbatim}
#include "camd.h"
long n, status, Ap [n+1], Ai [nz], P [n], C [n] ;
double Control [CAMD_CONTROL], Info [CAMD_INFO] ;
camd_l_defaults (Control) ;
status = camd_l_order (n, Ap, Ai, P, Control, Info, C) ;
camd_l_control (Control) ;
camd_l_info (Info) ;
status = camd_l_valid (n, n, Ap, Ai) ;
\end{verbatim}
}

%------------------------------------------------------------------------------
\section{Installation}
\label{Install}
%------------------------------------------------------------------------------

The following discussion assumes you have the {\tt make} program, either in
Unix, or in Windows with Cygwin.

System-dependent configurations are in the {\tt ../UFconfig/UFconfig.mk}
file.  You can edit that file to customize the compilation.  The default
settings will work on most systems.
Sample configuration files are provided
for Linux, Sun Solaris, SGI IRIX, IBM AIX, and the DEC/Compaq Alpha.

To compile and install the C-callable CAMD library,
go to the {\tt CAMD} directory and type {\tt make}.
The library will be placed in {\tt CAMD/Lib/libcamd.a}.
Three demo programs of the CAMD ordering routine will be compiled and tested in
the {\tt CAMD/Demo} directory.
The outputs of these demo programs will then be compared with output
files in the distribution.

Typing {\tt make clean} will remove all but the final compiled libraries
and demo programs.  Typing {\tt make purge} or {\tt make distclean}
removes all files not in the original distribution.
If you compile CAMD and then later change the {\tt ../UFconfig/UFconfig.mk} file
then you should type {\tt make purge} and then {\tt make} to recompile.

When you compile your program that uses the C-callable CAMD library,
you need to add the {\tt CAMD/Lib/libcamd.a} library
and you need to tell your compiler to look in the
{\tt CAMD/Include} directory for include
files.
See {\tt CAMD/Demo/Makefile} for an example.

If all you want to use is the CAMD mexFunction in MATLAB, you can skip
the use of the {\tt make} command entirely.  Simply type
{\tt camd\_make} in MATLAB while in the {\tt CAMD/MATLAB} directory.
This works on any system with MATLAB, including Windows.
Alternately, type {\tt make} in the {\tt CAMD/MATLAB} directory.

If you are including CAMD as a subset of a larger library and do not want
to link the C standard I/O library, or if you simply do not need to use
them, you can safely remove the {\tt camd\_control.c} and {\tt camd\_info.c}
files.  Similarly, if you use default parameters (or define your
own {\tt Control} array), then you can exclude the {\tt camd\_defaults.c}
file.
Each of these files contains the user-callable routines of the same
name.  None of these auxiliary routines are directly called by
{\tt camd\_order}.
The {\tt camd\_dump.c} file contains debugging routines
that are neither used nor compiled unless debugging is enabled.
The {\tt camd\_internal.h} file must be edited to enable debugging;
refer to the instructions in that file.
The bare minimum files required to use just {\tt camd\_order} are
{\tt camd.h} and {\tt camd\_internal.h}
in the {\tt Include} directory,
and
{\tt camd\_1.c},
{\tt camd\_2.c},
{\tt camd\_aat.c},
{\tt camd\_global.c},
{\tt and\_order.c},
{\tt camd\_postorder.c},
{\tt camd\_preprocess.c},
and
{\tt camd\_valid.c}
in the {\tt Source} directory.

%------------------------------------------------------------------------------
\newpage
\section{The CAMD routines}
\label{Primary}
%------------------------------------------------------------------------------

The file {\tt CAMD/Include/camd.h} listed below
describes each user-callable routine in CAMD,
and gives details on their use.

{\footnotesize
\begin{verbatim}
/* ========================================================================= */
/* === CAMD:  approximate minimum degree ordering ========================== */
/* ========================================================================= */

/* ------------------------------------------------------------------------- */
/* CAMD Version 2.1, Copyright (c) 2006 by Timothy A. Davis, Yanqing Chen,   */
/* Patrick R. Amestoy, and Iain S. Duff.  See ../README.txt for License.     */
/* email: davis at cise.ufl.edu    CISE Department, Univ. of Florida.        */
/* web: http://www.cise.ufl.edu/research/sparse/camd                         */
/* ------------------------------------------------------------------------- */

/* CAMD finds a symmetric ordering P of a matrix A so that the Cholesky
 * factorization of P*A*P' has fewer nonzeros and takes less work than the
 * Cholesky factorization of A.  If A is not symmetric, then it performs its
 * ordering on the matrix A+A'.  Two sets of user-callable routines are
 * provided, one for int integers and the other for UF_long integers.
 *
 * The method is based on the approximate minimum degree algorithm, discussed
 * in Amestoy, Davis, and Duff, "An approximate degree ordering algorithm",
 * SIAM Journal of Matrix Analysis and Applications, vol. 17, no. 4, pp.
 * 886-905, 1996.
 */

#ifndef CAMD_H
#define CAMD_H

/* make it easy for C++ programs to include CAMD */
#ifdef __cplusplus
extern "C" {
#endif

/* get the definition of size_t: */
#include <stddef.h>

/* define UF_long */
#include "UFconfig.h"

int camd_order              /* returns CAMD_OK, CAMD_OK_BUT_JUMBLED,
                             * CAMD_INVALID, or CAMD_OUT_OF_MEMORY */
(
    int n,                  /* A is n-by-n.  n must be >= 0. */
    const int Ap [ ],       /* column pointers for A, of size n+1 */
    const int Ai [ ],       /* row indices of A, of size nz = Ap [n] */
    int P [ ],              /* output permutation, of size n */
    double Control [ ],     /* input Control settings, of size CAMD_CONTROL */
    double Info [ ],        /* output Info statistics, of size CAMD_INFO */
    const int C [ ]         /* Constraint set of A, of size n; can be NULL */
) ;

UF_long camd_l_order        /* see above for description of arguments */
(
    UF_long n,
    const UF_long Ap [ ],
    const UF_long Ai [ ],
    UF_long P [ ],
    double Control [ ],
    double Info [ ],
    const UF_long C [ ]
) ;

/* Input arguments (not modified):
 *
 *      n: the matrix A is n-by-n.
 *      Ap: an int/UF_long array of size n+1, containing column pointers of A.
 *      Ai: an int/UF_long array of size nz, containing the row indices of A,
 *          where nz = Ap [n].
 *      Control:  a double array of size CAMD_CONTROL, containing control
 *          parameters.  Defaults are used if Control is NULL.
 *
 * Output arguments (not defined on input):
 *
 *      P: an int/UF_long array of size n, containing the output permutation. If
 *          row i is the kth pivot row, then P [k] = i.  In MATLAB notation,
 *          the reordered matrix is A (P,P).
 *      Info: a double array of size CAMD_INFO, containing statistical
 *          information.  Ignored if Info is NULL.
 *
 * On input, the matrix A is stored in column-oriented form.  The row indices
 * of nonzero entries in column j are stored in Ai [Ap [j] ... Ap [j+1]-1].
 *
 * If the row indices appear in ascending order in each column, and there
 * are no duplicate entries, then camd_order is slightly more efficient in
 * terms of time and memory usage.  If this condition does not hold, a copy
 * of the matrix is created (where these conditions do hold), and the copy is
 * ordered.
 *
 * Row indices must be in the range 0 to
 * n-1.  Ap [0] must be zero, and thus nz = Ap [n] is the number of nonzeros
 * in A.  The array Ap is of size n+1, and the array Ai is of size nz = Ap [n].
 * The matrix does not need to be symmetric, and the diagonal does not need to
 * be present (if diagonal entries are present, they are ignored except for
 * the output statistic Info [CAMD_NZDIAG]).  The arrays Ai and Ap are not
 * modified.  This form of the Ap and Ai arrays to represent the nonzero
 * pattern of the matrix A is the same as that used internally by MATLAB.
 * If you wish to use a more flexible input structure, please see the
 * umfpack_*_triplet_to_col routines in the UMFPACK package, at
 * http://www.cise.ufl.edu/research/sparse/umfpack.
 *
 * Restrictions:  n >= 0.  Ap [0] = 0.  Ap [j] <= Ap [j+1] for all j in the
 *      range 0 to n-1.  nz = Ap [n] >= 0.  Ai [0..nz-1] must be in the range 0
 *      to n-1.  Finally, Ai, Ap, and P must not be NULL.  If any of these
 *      restrictions are not met, CAMD returns CAMD_INVALID.
 *
 * CAMD returns:
 *
 *      CAMD_OK if the matrix is valid and sufficient memory can be allocated to
 *          perform the ordering.
 *
 *      CAMD_OUT_OF_MEMORY if not enough memory can be allocated.
 *
 *      CAMD_INVALID if the input arguments n, Ap, Ai are invalid, or if P is
 *          NULL.
 *
 *      CAMD_OK_BUT_JUMBLED if the matrix had unsorted columns, and/or duplicate
 *          entries, but was otherwise valid.
 *
 * The CAMD routine first forms the pattern of the matrix A+A', and then
 * computes a fill-reducing ordering, P.  If P [k] = i, then row/column i of
 * the original is the kth pivotal row.  In MATLAB notation, the permuted
 * matrix is A (P,P), except that 0-based indexing is used instead of the
 * 1-based indexing in MATLAB.
 *
 * The Control array is used to set various parameters for CAMD.  If a NULL
 * pointer is passed, default values are used.  The Control array is not
 * modified.
 *
 *      Control [CAMD_DENSE]:  controls the threshold for "dense" rows/columns.
 *          A dense row/column in A+A' can cause CAMD to spend a lot of time in
 *          ordering the matrix.  If Control [CAMD_DENSE] >= 0, rows/columns
 *          with more than Control [CAMD_DENSE] * sqrt (n) entries are ignored
 *          during the ordering, and placed last in the output order.  The
 *          default value of Control [CAMD_DENSE] is 10.  If negative, no
 *          rows/columns are treated as "dense".  Rows/columns with 16 or
 *          fewer off-diagonal entries are never considered "dense".
 *
 *      Control [CAMD_AGGRESSIVE]: controls whether or not to use aggressive
 *          absorption, in which a prior element is absorbed into the current
 *          element if is a subset of the current element, even if it is not
 *          adjacent to the current pivot element (refer to Amestoy, Davis,
 *          & Duff, 1996, for more details).  The default value is nonzero,
 *          which means to perform aggressive absorption.  This nearly always
 *          leads to a better ordering (because the approximate degrees are
 *          more accurate) and a lower execution time.  There are cases where
 *          it can lead to a slightly worse ordering, however.  To turn it off,
 *          set Control [CAMD_AGGRESSIVE] to 0.
 *
 *      Control [2..4] are not used in the current version, but may be used in
 *          future versions.
 *
 * The Info array provides statistics about the ordering on output.  If it is
 * not present, the statistics are not returned.  This is not an error
 * condition.
 * 
 *      Info [CAMD_STATUS]:  the return value of CAMD, either CAMD_OK,
 *          CAMD_OK_BUT_JUMBLED, CAMD_OUT_OF_MEMORY, or CAMD_INVALID.
 *
 *      Info [CAMD_N]: n, the size of the input matrix
 *
 *      Info [CAMD_NZ]: the number of nonzeros in A, nz = Ap [n]
 *
 *      Info [CAMD_SYMMETRY]:  the symmetry of the matrix A.  It is the number
 *          of "matched" off-diagonal entries divided by the total number of
 *          off-diagonal entries.  An entry A(i,j) is matched if A(j,i) is also
 *          an entry, for any pair (i,j) for which i != j.  In MATLAB notation,
 *              S = spones (A) ;
 *              B = tril (S, -1) + triu (S, 1) ;
 *              symmetry = nnz (B & B') / nnz (B) ;
 *
 *      Info [CAMD_NZDIAG]: the number of entries on the diagonal of A.
 *
 *      Info [CAMD_NZ_A_PLUS_AT]:  the number of nonzeros in A+A', excluding the
 *          diagonal.  If A is perfectly symmetric (Info [CAMD_SYMMETRY] = 1)
 *          with a fully nonzero diagonal, then Info [CAMD_NZ_A_PLUS_AT] = nz-n
 *          (the smallest possible value).  If A is perfectly unsymmetric
 *          (Info [CAMD_SYMMETRY] = 0, for an upper triangular matrix, for
 *          example) with no diagonal, then Info [CAMD_NZ_A_PLUS_AT] = 2*nz
 *          (the largest possible value).
 *
 *      Info [CAMD_NDENSE]: the number of "dense" rows/columns of A+A' that were
 *          removed from A prior to ordering.  These are placed last in the
 *          output order P.
 *
 *      Info [CAMD_MEMORY]: the amount of memory used by CAMD, in bytes.  In the
 *          current version, this is 1.2 * Info  [CAMD_NZ_A_PLUS_AT] + 9*n
 *          times the size of an integer.  This is at most 2.4nz + 9n.  This
 *          excludes the size of the input arguments Ai, Ap, and P, which have
 *          a total size of nz + 2*n + 1 integers.
 *
 *      Info [CAMD_NCMPA]: the number of garbage collections performed.
 *
 *      Info [CAMD_LNZ]: the number of nonzeros in L (excluding the diagonal).
 *          This is a slight upper bound because mass elimination is combined
 *          with the approximate degree update.  It is a rough upper bound if
 *          there are many "dense" rows/columns.  The rest of the statistics,
 *          below, are also slight or rough upper bounds, for the same reasons.
 *          The post-ordering of the assembly tree might also not exactly
 *          correspond to a true elimination tree postordering.
 *
 *      Info [CAMD_NDIV]: the number of divide operations for a subsequent LDL'
 *          or LU factorization of the permuted matrix A (P,P).
 *
 *      Info [CAMD_NMULTSUBS_LDL]:  the number of multiply-subtract pairs for a
 *          subsequent LDL' factorization of A (P,P).
 *
 *      Info [CAMD_NMULTSUBS_LU]:  the number of multiply-subtract pairs for a
 *          subsequent LU factorization of A (P,P), assuming that no numerical
 *          pivoting is required.
 *
 *      Info [CAMD_DMAX]:  the maximum number of nonzeros in any column of L,
 *          including the diagonal.
 *
 *      Info [14..19] are not used in the current version, but may be used in
 *          future versions.
 */    

/* ------------------------------------------------------------------------- */
/* direct interface to CAMD */
/* ------------------------------------------------------------------------- */

/* camd_2 is the primary CAMD ordering routine.  It is not meant to be
 * user-callable because of its restrictive inputs and because it destroys
 * the user's input matrix.  It does not check its inputs for errors, either.
 * However, if you can work with these restrictions it can be faster than
 * camd_order and use less memory (assuming that you can create your own copy
 * of the matrix for CAMD to destroy).  Refer to CAMD/Source/camd_2.c for a
 * description of each parameter. */

void camd_2
(
    int n,
    int Pe [ ],
    int Iw [ ],
    int Len [ ],
    int iwlen,
    int pfree,
    int Nv [ ],
    int Next [ ], 
    int Last [ ],
    int Head [ ],
    int Elen [ ],
    int Degree [ ],
    int W [ ],
    double Control [ ],
    double Info [ ],
    const int C [ ],
    int BucketSet [ ] 
) ;

void camd_l2
(
    UF_long n,
    UF_long Pe [ ],
    UF_long Iw [ ],
    UF_long Len [ ],
    UF_long iwlen,
    UF_long pfree,
    UF_long Nv [ ],
    UF_long Next [ ], 
    UF_long Last [ ],
    UF_long Head [ ],
    UF_long Elen [ ],
    UF_long Degree [ ],
    UF_long W [ ],
    double Control [ ],
    double Info [ ],
    const UF_long C [ ],
    UF_long BucketSet [ ]
    
) ;

/* ------------------------------------------------------------------------- */
/* camd_valid */
/* ------------------------------------------------------------------------- */

/* Returns CAMD_OK or CAMD_OK_BUT_JUMBLED if the matrix is valid as input to
 * camd_order; the latter is returned if the matrix has unsorted and/or
 * duplicate row indices in one or more columns.  Returns CAMD_INVALID if the
 * matrix cannot be passed to camd_order.  For camd_order, the matrix must also
 * be square.  The first two arguments are the number of rows and the number
 * of columns of the matrix.  For its use in CAMD, these must both equal n.
 */

int camd_valid
(
    int n_row,              /* # of rows */
    int n_col,              /* # of columns */
    const int Ap [ ],       /* column pointers, of size n_col+1 */
    const int Ai [ ]        /* row indices, of size Ap [n_col] */
) ;

UF_long camd_l_valid
(
    UF_long n_row,
    UF_long n_col,
    const UF_long Ap [ ],
    const UF_long Ai [ ]
) ;

/* ------------------------------------------------------------------------- */
/* camd_cvalid */
/* ------------------------------------------------------------------------- */

/* Returns TRUE if the constraint set is valid as input to camd_order,
 * FALSE otherwise. */

int camd_cvalid
(
   int n,
   const int C [ ]
) ;

UF_long camd_l_cvalid
(
   UF_long n,
   const UF_long C [ ]
) ;

/* ------------------------------------------------------------------------- */
/* CAMD memory manager and printf routines */
/* ------------------------------------------------------------------------- */

/* The user can redefine these to change the malloc, free, and printf routines
 * that CAMD uses. */

#ifndef EXTERN
#define EXTERN extern
#endif

EXTERN void *(*camd_malloc) (size_t) ;              /* pointer to malloc */
EXTERN void (*camd_free) (void *) ;                 /* pointer to free */
EXTERN void *(*camd_realloc) (void *, size_t) ;     /* pointer to realloc */
EXTERN void *(*camd_calloc) (size_t, size_t) ;      /* pointer to calloc */
EXTERN int (*camd_printf) (const char *, ...) ;     /* pointer to printf */

/* ------------------------------------------------------------------------- */
/* CAMD Control and Info arrays */
/* ------------------------------------------------------------------------- */

/* camd_defaults:  sets the default control settings */
void camd_defaults   (double Control [ ]) ;
void camd_l_defaults (double Control [ ]) ;

/* camd_control: prints the control settings */
void camd_control    (double Control [ ]) ;
void camd_l_control  (double Control [ ]) ;

/* camd_info: prints the statistics */
void camd_info       (double Info [ ]) ;
void camd_l_info     (double Info [ ]) ;

#define CAMD_CONTROL 5      /* size of Control array */
#define CAMD_INFO 20        /* size of Info array */

/* contents of Control */
#define CAMD_DENSE 0        /* "dense" if degree > Control [0] * sqrt (n) */
#define CAMD_AGGRESSIVE 1    /* do aggressive absorption if Control [1] != 0 */

/* default Control settings */
#define CAMD_DEFAULT_DENSE 10.0     /* default "dense" degree 10*sqrt(n) */
#define CAMD_DEFAULT_AGGRESSIVE 1    /* do aggressive absorption by default */

/* contents of Info */
#define CAMD_STATUS 0       /* return value of camd_order and camd_l_order */
#define CAMD_N 1                    /* A is n-by-n */
#define CAMD_NZ 2           /* number of nonzeros in A */ 
#define CAMD_SYMMETRY 3     /* symmetry of pattern (1 is sym., 0 is unsym.) */
#define CAMD_NZDIAG 4       /* # of entries on diagonal */
#define CAMD_NZ_A_PLUS_AT 5  /* nz in A+A' */
#define CAMD_NDENSE 6       /* number of "dense" rows/columns in A */
#define CAMD_MEMORY 7       /* amount of memory used by CAMD */
#define CAMD_NCMPA 8        /* number of garbage collections in CAMD */
#define CAMD_LNZ 9          /* approx. nz in L, excluding the diagonal */
#define CAMD_NDIV 10        /* number of fl. point divides for LU and LDL' */
#define CAMD_NMULTSUBS_LDL 11 /* number of fl. point (*,-) pairs for LDL' */
#define CAMD_NMULTSUBS_LU 12  /* number of fl. point (*,-) pairs for LU */
#define CAMD_DMAX 13         /* max nz. in any column of L, incl. diagonal */

/* ------------------------------------------------------------------------- */
/* return values of CAMD */
/* ------------------------------------------------------------------------- */

#define CAMD_OK 0               /* success */
#define CAMD_OUT_OF_MEMORY -1   /* malloc failed, or problem too large */
#define CAMD_INVALID -2         /* input arguments are not valid */
#define CAMD_OK_BUT_JUMBLED 1   /* input matrix is OK for camd_order, but
    * columns were not sorted, and/or duplicate entries were present.  CAMD had
    * to do extra work before ordering the matrix.  This is a warning, not an
    * error.  */

/* ========================================================================== */
/* === CAMD version ========================================================= */
/* ========================================================================== */

/*
 * As an example, to test if the version you are using is 1.2 or later:
 *
 *      if (CAMD_VERSION >= CAMD_VERSION_CODE (1,2)) ...
 *
 * This also works during compile-time:
 *
 *      #if (CAMD_VERSION >= CAMD_VERSION_CODE (1,2))
 *          printf ("This is version 1.2 or later\n") ;
 *      #else
 *          printf ("This is an early version\n") ;
 *      #endif
 */

#define CAMD_DATE "June 27, 2006"
#define CAMD_VERSION_CODE(main,sub) ((main) * 1000 + (sub))
#define CAMD_MAIN_VERSION 2
#define CAMD_SUB_VERSION 1
#define CAMD_VERSION CAMD_VERSION_CODE(CAMD_MAIN_VERSION,CAMD_SUB_VERSION)

#ifdef __cplusplus
}
#endif

#endif
\end{verbatim}
}

%------------------------------------------------------------------------------
\newpage
% References
%------------------------------------------------------------------------------

\bibliographystyle{plain}
\bibliography{CAMD_UserGuide}

\end{document}
